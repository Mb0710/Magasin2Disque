\documentclass{beamer}

% --- Packages ---
\usepackage[utf8]{inputenc}
\usepackage[T1]{fontenc}
\usepackage[french]{babel}
\usepackage{graphicx}
\usepackage{booktabs}
\usepackage{tikz} % Indispensable pour les dégradés

% --- Thème et Couleurs ---
\usetheme{Madrid}

% 1. Définition des couleurs
\definecolor{myBlue}{RGB}{20, 40, 160}   % Bleu profond
\definecolor{myViolet}{RGB}{140, 20, 160} % Violet

% 2. Configuration de base
\usecolortheme{whale}
\setbeamercolor{structure}{fg=myBlue}
\setbeamercolor{title}{fg=white} % Texte du titre en blanc
\setbeamercolor{frametitle}{fg=white}

% --- CUSTOMISATION 1 : BARRE DE TITRE DES SLIDES ---
\setbeamertemplate{frametitle}{
    \nointerlineskip
    \begin{beamercolorbox}[wd=\paperwidth,ht=1.2cm]{frametitle}
        \begin{tikzpicture}[remember picture, overlay]
            \shade[left color=myBlue, right color=myViolet, shading angle=45] 
            (0,0) rectangle (\paperwidth,1.2cm);
            \node[anchor=west, white, font=\large\bfseries] at (0.5,0.61) {\insertframetitle};
        \end{tikzpicture}
    \end{beamercolorbox}
}

% --- CUSTOMISATION 2 : PAGE DE TITRE (Titre + Sous-titre en dégradé) ---
\setbeamertemplate{title page}{
    \vspace{1cm} % Marge haut
    \centering
    
    % Boîte dégradée pour le TITRE et SOUS-TITRE
    \begin{tikzpicture}
        \node[
            shade, left color=myBlue, right color=myViolet, shading angle=45, % Le dégradé
            text width=0.9\paperwidth, % Largeur de la boite
            inner sep=10pt, % Espace intérieur
            rounded corners=5pt, % Bords arrondis
            align=center,
            text=white % Couleur du texte
        ] 
        {
            \LARGE \bfseries \inserttitle \\[0.5em] 
            \large \insertsubtitle
        };
    \end{tikzpicture}
    
    \vspace{1cm}
    
    % Auteurs (Liste des membres)
    {\usebeamerfont{author}\insertauthor} \par
    \vspace{0.5cm}
    
    % Institution (École et Groupe)
    {\usebeamerfont{institute}\insertinstitute} \par
    \vspace{1cm}
    
    % Date
    {\usebeamerfont{date}\insertdate} \par
    \vspace{0.5cm}
}

% --- CUSTOMISATION 3 : BLOCS (Header en dégradé) ---
\setbeamertemplate{block begin}{
    \par\vskip\medskipamount
    % Le header du bloc avec dégradé
    \begin{tikzpicture}
        \node[
            shade, left color=myBlue, right color=myViolet, shading angle=45,
            text width=\linewidth-10pt, % Ajustement largeur
            inner sep=5pt,
            rounded corners=2pt,
            text=white,
            font=\bfseries
        ] (box) {\insertblocktitle};
    \end{tikzpicture}
    \par
    % Le corps du bloc (gris clair standard)
    \usebeamerfont{block body}
    \begin{beamercolorbox}[vmode, colsep*=.75ex]{block body}
}

\setbeamertemplate{block end}{
    \end{beamercolorbox}
    \vskip\smallskipamount
}


% --- Méta-données (MISES À JOUR ICI) ---
\title[Framework Actor]{Projet de framework d'acteurs coopérant via microservices }
\subtitle{Application : Marketplace de Vinyles "Magasin2Disque"}

% Noms des membres
\author[Groupe 4]{
    BENLAIFAOUI Mohamed \\ 
    LAID Yanis \\ 
    MOUCHARD Damien \\ 
    SOUSSA Wassim
}

% École et Groupe
\institute[CY Tech]{
    CY Tech \\
    ING2 GSI groupe 4
}

% Date mise à jour
\date{9 décembre 2025}

\begin{document}

% --- Slide de Titre ---
\begin{frame}
    \titlepage 
\end{frame}

% --- Plan ---
\begin{frame}{Sommaire}
    \tableofcontents
\end{frame}

% ==========================================
% PARTIE 1 : CONTEXTE ET OBJECTIFS
% ==========================================
\section{Contexte et Objectifs}

\begin{frame}{Objectifs du Projet}
    \begin{block}{Le Besoin}
        Développer un framework inspiré d'\textbf{Akka} permettant la coopération d'entités autonomes (acteurs) au sein d'une architecture microservices \textbf{Spring Boot}.
    \end{block}

    \vspace{0.5cm}
    
    \textbf{Les 3 Piliers du Système :}
    \begin{itemize}
        \setlength\itemsep{0.5em}
        \item \textbf{Résilience :} Tolérance aux pannes grâce à la supervision hiérarchique.
        \item \textbf{Élasticité :} Adaptation dynamique à la charge.
        \item \textbf{Communication :} Échanges asynchrones (messages).
    \end{itemize}
\end{frame}

% ==========================================
% PARTIE 2 : ARCHITECTURE
% ==========================================
\section{Architecture du Framework}

\begin{frame}{Architecture générale}
    
    \begin{block}{Démarche Technique}
        Afin de comprendre le fonctionnement interne de notre solution, nous allons explorer l'architecture à travers \textbf{6 vues techniques} successives, allant de la vision globale aux mécanismes de résilience.
    \end{block}

    \vspace{0.5cm}

    \begin{enumerate}
        \setlength\itemsep{1em} % Espacement entre les points
        \item \textbf{Vue d'ensemble :} Interaction globale des composants (Front, Gateway, Registry).
        \item \textbf{Cœur du Système :} Architecture interne du \textit{Framework Actor} (Système, Scheduler, EventBus).
        \item \textbf{Point d'Entrée (Gateway) :} Gestion du routage dynamique et de la sécurité.
        \item \textbf{Communication Inter-Services :} Le pattern Proxy avec OpenFeign.
        \item \textbf{Résilience :} Tolérance aux pannes via le pattern \textit{Circuit Breaker}.
        \item \textbf{Persistance :} Ségrégation des données (\textit{Database per Service}) avec JPA.
    \end{enumerate}

\end{frame}

\begin{frame}{1. Vue d'ensemble de l'Architecture}
    \begin{figure}
        \centering
        \includegraphics[height=0.75\textheight, keepaspectratio]{1Vue d'ensemble de l'Architecture PRESENTATION.jpg}
    \end{figure}
\end{frame}

\begin{frame}{2. API Gateway}
    \begin{figure}
        \centering
        \includegraphics[height=0.75\textheight, keepaspectratio]{4API Gateway PRESENTATION.jpg}
    \end{figure}
\end{frame}

\begin{frame}{3. OpenFeign}
    \begin{figure}
        \centering
        \includegraphics[height=0.75\textheight, keepaspectratio]{5OpenFeignPRESENTATION.jpg}
    \end{figure}
\end{frame}

\begin{frame}{4. Resilience4j}
    \begin{figure}
        \centering
        \includegraphics[height=0.75\textheight, keepaspectratio]{6Resilience4j (Circuit Breaker)PRESENTATION.jpg}
    \end{figure}
\end{frame}

\begin{frame}{5. Spring Data JPA et PostgreSQL}
    \begin{figure}
        \centering
        \includegraphics[height=0.75\textheight, keepaspectratio]{8Spring Data JPA_PostgreSQLpresentation.jpg}
    \end{figure}
\end{frame}

% CORRECTION DE L'ERREUR CRITIQUE ICI (double accolade {{ retirée)
\begin{frame}{Architecture du Framework}
    
    \begin{alertblock}{La Problématique}
        Comment gérer la concurrence et l'état interne des services sans utiliser les blocages complexes des Threads Java classiques ?
    \end{alertblock}

    \vspace{0.8cm}

    \textbf{Notre Solution : Une implémentation du modèle Actor}
    
    Nous avons développé notre propre moteur inspiré d'Akka. Nous allons maintenant détailler son fonctionnement en 3 étapes clés :

    \vspace{0.5cm}

    \begin{itemize}
        \setlength\itemsep{1em}
        \item[\textbf{1.}] \textbf{L'Orchestration :} Comment l'\textit{ActorSystem} gère l'univers des acteurs.
        \item[\textbf{2.}] \textbf{L'Isolation :} Le couple \textit{Actor / ActorRef} pour l'asynchronisme total.
        \item[\textbf{3.}] \textbf{L'Auto-Réparation :} La hiérarchie de \textit{Supervision} pour la résilience.
    \end{itemize}

\end{frame}

\begin{frame}{1. Vue d'ensemble de l'Architecture}
    \begin{figure}
        \centering
        \includegraphics[height=0.75\textheight, keepaspectratio]{1Architecture générale FE.jpg}
    \end{figure}
\end{frame}

\begin{frame}{2. Composants principaux : ActorSystem}
    \begin{figure}
        \centering
        \includegraphics[height=0.75\textheight, keepaspectratio]{ActorSystem FE.jpg}
    \end{figure}
\end{frame}

\begin{frame}{2. Composants principaux : Actor}
    \begin{figure}
        \centering
        \includegraphics[height=0.75\textheight, keepaspectratio]{Actor(interface) FE.jpg}
    \end{figure}
\end{frame}

\begin{frame}{3. Composants principaux : ActorRef}
    \begin{figure}
        \centering
        \includegraphics[height=0.75\textheight, keepaspectratio]{ActorRef(interface)FE.jpg}
    \end{figure}
\end{frame}

\begin{frame}{4. Composants principaux : Supervision Strategy}
    \begin{figure}
        \centering
        \includegraphics[height=0.75\textheight, keepaspectratio]{SupervisionStrategy(interface)FE.jpg}
    \end{figure}
\end{frame}



% ==========================================
% PARTIE 3 : SCÉNARIO (Images)
% ==========================================
\section{Démonstration}

\begin{frame}{Scénario : Alice vend l'album "Abbey Road" à Bob}
    \begin{enumerate}
        \item Alice s'inscrit et se connecte.
        \item Alice crée une annonce pour vendre "Abbey Road" des Beatles à 25€.
        \item Bob consulte les annonces et voit celle d'Alice.
        \item Bob fait une offre de 20€.
        \item Alice reçoit une notification 
        \item Elle accepte l'offre.  Une transaction est créée automatiquement.
        \item Alice envoie le disque et confirme la livraison.
        \item Bob reçoit le disque et confirme la réception.$\rightarrow$  La transaction est marquée comme "DELIVERED".
    \end{enumerate}
\end{frame}

\begin{frame}{1. Interface de Création de compte}
    \begin{figure}
        \centering
        \includegraphics[height=0.75\textheight, keepaspectratio]{page_création_compte.png}
    \end{figure}
\end{frame}

\begin{frame}{2. Interface de Connexion}
    \begin{figure}
        \centering
        \includegraphics[height=0.75\textheight, keepaspectratio]{page_connexion.png}
    \end{figure}
\end{frame}

\begin{frame}{3. Création de l'annonce}
    \begin{figure}
        \centering
        \includegraphics[height=0.75\textheight, keepaspectratio]{creation_annonce.png}
        \caption{Formulaire géré par \texttt{AnnonceActor}.}
    \end{figure}
\end{frame}

\begin{frame}{4. Page d'accueil}
    \begin{figure}
        \centering
        \includegraphics[height=0.75\textheight, keepaspectratio]{page_accueil.png}
    \end{figure}
\end{frame}

\begin{frame}{5. Page de chat}
    \begin{figure}
        \centering
        \includegraphics[height=0.75\textheight, keepaspectratio]{conversation_avec_autre_user.png}
    \end{figure}
\end{frame}

\begin{frame}{6. Page de notifications}
    \begin{figure}
        \centering
        \includegraphics[height=0.75\textheight, keepaspectratio]{page_notif.png}
    \end{figure}
\end{frame}

\begin{frame}{7. Page de Transaction }
    \begin{figure}
        \centering
        \includegraphics[height=0.75\textheight, keepaspectratio]{page_transaction.png}
        \caption{L'état passe à "DELIVERED".}
    \end{figure}
\end{frame}

\begin{frame}{8.a. Confirmation pour Alice }
    \begin{figure}
        \centering
        \includegraphics[height=0.75\textheight, keepaspectratio]{mail_confirmation.png}
        \caption{L'état passe à "DELIVERED".}
    \end{figure}
\end{frame}

\begin{frame}{8.b. Confirmation pour Bob }
    \begin{figure}
        \centering
        \includegraphics[height=0.75\textheight, keepaspectratio]{mail_confirmation2.png}
        \caption{L'état passe à "DELIVERED".}
    \end{figure}
\end{frame}

\begin{frame}{1/3 Dashboard admin : Gestion des utilisateurs}
    \begin{figure}
        \centering
        \includegraphics[height=0.75\textheight, keepaspectratio]{gestion_utilisateur.png}
        \caption{L'état passe à "DELIVERED".}
    \end{figure}
\end{frame}

\begin{frame}{2/3 Dashboard admin : Gestion des utilisateurs}
    \begin{figure}
        \centering
        \includegraphics[height=0.75\textheight, keepaspectratio]{gestion_utilisateur2.png}
        \caption{L'état passe à "DELIVERED".}
    \end{figure}
\end{frame}

\begin{frame}{3/3 Dashboard admin : Gestion des annonces}
    \begin{figure}
        \centering
        \includegraphics[height=0.75\textheight, keepaspectratio]{gestion_annonce.png}
        \caption{L'état passe à "DELIVERED".}
    \end{figure}
\end{frame}

\begin{frame}{Stack Technologique}
    \begin{table}
        \centering
        \footnotesize % Réduction de la taille pour que tout rentre
        \setlength{\tabcolsep}{8pt} % Espacement horizontal entre les colonnes
        \renewcommand{\arraystretch}{1.2} % Espacement vertical entre les lignes
        
        \begin{tabular}{lll}
            \toprule
            \textbf{Technologie} & \textbf{Rôle} & \textbf{Avantage Principal} \\
            \midrule
            \textbf{Spring Boot} & Framework backend & Configuration automatique \\
            \textbf{Spring Cloud} & Microservices & Ecosystème complet \\
            \textbf{Eureka} & Service Discovery & Auto-enregistrement \\
            \textbf{API Gateway} & Point d'entrée & Routage dynamique \\
            \textbf{OpenFeign} & HTTP Client & Simplicité d'utilisation \\
            \textbf{Resilience4j} & Circuit Breaker & Résilience \\
            \textbf{Spring Security} & Sécurité & Protection endpoints \\
            \textbf{JWT} & Authentification & Stateless \\
            \textbf{Spring Data JPA} & ORM & Abstraction DB \\
            \textbf{PostgreSQL} & Base de données & Fiabilité \\
            \textbf{Actor Framework} & Messaging & Asynchrone \\
            \textbf{Vanilla JS} & Frontend & Léger et rapide \\
            \bottomrule
        \end{tabular}
        \caption{Résumé des technologies employées}
    \end{table}
\end{frame}

% ==========================================
% CONCLUSION
% ==========================================
\section{Conclusion}

\begin{frame}{Bilan du Projet}
    \begin{block}{Réalisations}
        \begin{itemize}
            \item Framework Actor fonctionnel.
            \item Architecture microservices découplée.
            \item Résilience et isolation des pannes.
        \end{itemize}
    \end{block}
    
    \vspace{1cm}
    \centering
    \Large \textbf{Merci de votre attention !}
\end{frame}

\end{document}
